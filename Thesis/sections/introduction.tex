Even for the most experienced drivers, driving any vehicle is a difficult task. This difficulty can be shown by the high number of collisions in Canada alone. In 2015, there were 118,404 collisions that were either fatal or involved a personal injury~\cite{TrafficReport}. In about 84\% of all accidents the cause was traced back to driver error~\cite{CognitiveModel}. 

Although the majority of vehicles are currently equipped with passive safety systems, \emph{i.e.} systems to help reduce the outcome of an accident, such as seat belts, airbags, \emph{etc.}, there are still a high number of serious accidents. Newer intelligent car models are becoming equipped with active safety systems that utilize an understanding of the vehicle's state to avoid and minimize the effects of a crash. Some of these systems include collision warning and adaptive cruise control. Research into these active safety systems have expanded into applications that work with or for the driver. This new generation of driver-assistance systems go beyond automated control systems by attempting to work in combination with a driver. These advanced safety systems include predicting driver intent~\cite{InferringDriverIntent:2004}, warning drivers of lane departures~\cite{LaneDeparture:2002}, \emph{etc}. Although these active systems have many benefits, they are difficult to implement as they require knowledge about the driver, the vehicle, or the environment.

Furthermore, as these new technologies becoming embedded into users' everyday life, drivers need to make sure they are paying adequate attention to their current driving environment. To this end, there has been numerous works on modeling and monitoring drivers' attentiveness. Many of these works attempt to directly correlate driver attention to other measurements such as drowsiness, head movement and position, or alertness~\cite{VisualCuesDriverVigilance:2001, DrowsinessWarning:2008, DAISY:1994, Drowsiness:2012}. Many of these systems, often referred to as driver attention monitoring systems, work sufficiently well leading to newer vehicles becoming equipped with them. These vehicle safety systems were first introduced by Toyota in 2006 for its latest Lexus models~\cite{Toyota:2015}. Their system uses infrared sensors to monitor the driver's attention level. Specifically, these driver monitoring systems include a camera placed on the steering column which is capable of eye tracking via infrared LED detectors. In the case that the driver is not paying adequate attention to the road ahead, and a dangerous situation is detected, the system will warn the driver by flashing lights or providing warning sounds. If no action is taken by the driver, the vehicle will enter automation mode and apply the brakes (a warning alarm will sound followed by a brief automatic application of the braking system). After Toyota entered the market, various other companies began to follow their lead.

BMW started to offer an Active Driving Assistant with Attention model that analyses the user's driving behavior and, if necessary, advises the driver to rest. The alert system notifies the driver as to when to take a break in the form of graphic symbols shown on the control display~\cite{bmw}. Similarly, Bosch offers a driver drowsiness detection system that takes input from a combination of steering angle sensors, a front-mounted lane assisting camera, vehicle speed, and turn signal information~\cite{bosch}. Using this information, a driver drowsiness level is computed. Additionally, Ford, Hyundai, and Kia all come fully installed with their respective driver attention warning systems. These systems were first debuted by Ford with their 2011 Ford Focus~\cite{ford}, Hyundai on their 2017 i30, and lastly, Kia with their 2018 Stinger.

Although these systems can provide a broad understanding of the driver's attentiveness, they address the driver's performance as opposed to the driver's need. Additionally, many of these systems require a difficult installation and calibration process. To this end, the proposed research aims to gather knowledge related to the driver's required attention level from a single outward-facing dashcam. To accomplish this, dashcam videos are processed by a two-stream recurrent convolutional architecture to produce a label corresponding to the driver's required attention level in each driving situation. The attention levels are divided into four categories: low attention, medium attention, high attention, and very high attention.

The proposed two-stream recurrent convolutional architecture is based on a two-stream approach that models both spatial and temporal features in separate convolutional neural network (CNN) streams. These streams are then combined and fed into a Recurrent Neural Network (RNN) that models the new stream of features as hidden units in time. Both streams are built upon a pre-trained VGG~\cite{VGGNet:2015} network that is fine-tuned with the newly annotated attention dataset. The spatial stream takes still video frames as inputs and outputs a stream of high-level appearance features. The temporal stream utilizes simple flow fields extracted from pairs of adjacent frames and outputs high-level motion features. The final output features of both streams are then combined and fed into a RNN using Long Short-Term Memory (LSTM) units~\cite{LSTM:1997}. Since the proposed algorithm assigns a single attention level to the full video sequence, only the final hidden state of the RNN model is used. This last hidden state is then feed into a Softmax classifier to produce the final output of the network, \emph{i.e.} the driver's required attention level. The complete process runs in real-time (ten frames per second) on a laptop GPU.

In particular, the main contributions of this paper are as follows: 1) create a novel attention dataset that addresses the driver's need as opposed to the driver's performance, 2) propose a two-stream recurrent convolutional architecture that estimates the driver's required attention level via outward-facing dashcam videos in real-time, and 3) help improve driver safety by proposing a system that can analyze the road ahead and provide intuitive feedback to the driver.

The rest of this paper is organized as follows. Section 2 presents related work on driver attention that employs various methods use as gaze following using convnets. The explanation of the structure and training scheme for the two-stream recurrent convolutional architecture is presented in Section 3 and 4 respectively. Section 5 outlines the experiments and presents the prediction performance of the proposed model. Future works and concluding remarks are provided in Section 6.


\section{Objectives and Scope}
The main objectives of this research are:
\begin{itemize}
  \item Determine a multiple neural network architecture that can learn to estimate a driver's required attention levels in various driving situations.
  \item Train the network on a newly labelled dataset containing annotations of the driver's perceived attention levels for each driving situation.
  \item Run the algorithm in real-time using only a single, outward facing dashcam as input.
\end{itemize}

\section{Overview}
This thesis consists of (insert number of chapters) chapters. A short summary of each chapter is provided in the following section. The chapter order reflects the manner in which this research program was evolved.

Chapter 2 describes the fundamental framework used to develop this research program. Among the topics presented are recent video classification algorithms, the extension into vehicles for vehicle intent, and lastly models for driver attention.

Chapter 3 describes...

